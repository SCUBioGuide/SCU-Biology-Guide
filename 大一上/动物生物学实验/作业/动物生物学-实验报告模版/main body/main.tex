\documentclass[a4paper,10pt]{article}

\usepackage[UTF8, heading=true]{ctex}
\usepackage{xeCJK}
\usepackage{mathptmx}
\usepackage{anyfontsize}
\usepackage{t1enc}
\usepackage{graphicx}
\usepackage[top=20mm,bottom=20mm,left=20mm,right=20mm]{geometry}
\usepackage{wrapfig}
\usepackage{float} %设置图片浮动位置的宏包
\usepackage{subfigure} %插入多图时用子图显示的宏包
\usepackage[justification=centering]{caption}


\newcommand{\upcite}[1]{\textsuperscript{\textsuperscript{\cite{#1}}}} 

\makeatletter
  \g@addto@macro{\CTEX@section@format}{\raggedright}
\makeatother

\makeatletter
  \def\vhrulefill#1{\leavevmode\leaders\hrule\@height#1\hfill \kern\z@}
\makeatother

\setCJKmainfont{Heiti SC} 
\pagestyle{plain}

\begin{document}
  % 标题区域开始
  \begin{center}
    \zihao{2}四川大学实验报告 \\
    \vspace{1em}
    \begin{tabular}{lll}
      \zihao{4}学\hspace{1em}院\hspace{1em}生命科学院 & \zihao{4}专\hspace{2em}业\hspace{1em}生物科学\hspace{2em} & \zihao{4}\hspace{2em}级\hspace{2em}班\hspace{1em}组\\
      
      \zihao{4}姓\hspace{1em}名\hspace{1em}\hspace{3em} & \zihao{4}同实验者\hspace{1em}\hspace{3em} & \zihao{4}\hspace{1em}年\hspace{1em}月\hspace{1em}日
    \end{tabular}

    \vhrulefill{2pt}
  \end{center}
  \zihao{4}题\hspace{1em}目:
  % 标题区域结束

  \section{实验目的}

  \section{实验原理}
   
  \section{实验步骤}

  \section{实验结果}
  
  \section{讨论}
   



  \clearpage
  \begin{thebibliography}{999}
    
  \end{thebibliography}

\end{document}